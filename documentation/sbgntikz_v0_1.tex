\documentclass{scrartcl}

\usepackage{makeidx}
\usepackage[cache]{minted}
\usepackage{tikz}
\usepackage{colortbl}
\usepackage{mfirstuc}
\usepackage{tabu}
\usepackage{fancyvrb}

% \usepackage{etoolbox}
% \let\originalinput\input
% \newcommand{\newinput}[1]{\InputIfFileExists{#1}{}{}}
% \AtBeginEnvironment{tabu*}{\let\input\newinput}

\usetikzlibrary{positioning, calc, sbgn}

\def\texttikz{Ti\textit{k}Z}
\def\textpgf{\texttt{PGF}}
\def\tikzmanual{\textpgf/\texttikz{} manual}
\def\sbgntikz{SBGN\texttikz}

\def\colkey#1{
    \texttt{\textcolor{red!80}{#1}}
}

% \newenvironment{entry}[3]
% {
%     \glyph{#1}{#2}{#3}
%     \par
%     \leftskip10pt
%     \begingroup
% }
% {
%     \endgroup
%     \par
%     \vspace{0.5cm}
% }

\newenvironment{entry}[3]
{
    \noindent\glyph{#1}{#2}{#3}
    \begin{flushright}
    \begin{minipage}[c]{0.95\linewidth}
}
{
    \end{minipage}
    \end{flushright}
    % \vspace{0.5cm}
}

\def\glyph#1#2#3{
    \index{\MakeLowercase #1}
    \large\textsf{#1 (#3)}\normalsize \hfill \colkey{/tikz/#2}
}

% Shamelessly stolen from minted manual's source code ;)
\newenvironment{code}
{
    \vskip5pt
    \VerbatimEnvironment
    \begin{VerbatimOut}{code.out}%
}
{
    \end{VerbatimOut}%
    \colorbox{blue!10}{
        \begin{minipage}[c]{0.23\textwidth}
        \center
        \begin{tikzpicture}[sbgn]
            \input{code.out}
        \end{tikzpicture}
    \end{minipage}}
    \colorbox{gray!10}{
        \begin{minipage}[c]{0.70\textwidth}%
        \inputminted[resetmargins, breaklines]{latex}{code.out}%
    \end{minipage}}
    \vskip5pt
}

\title{sbgntikz}
\subtitle{manual for version 0.1}
\author{Adrien Rougny}

\makeindex

\begin{document}

\maketitle

\begin{center}
\scalebox{0.47}{
    \begin{tikzpicture}[sbgn,
        node distance = 1.5cm,
        simple chemical/.append style={fill=blue!40},
        complex/.append style={fill=red!40},
        macromolecule/.append style={fill=orange!40},
        phenotype/.append style={fill=green!40},
        submap/.append style={fill=yellow!40},
        compartment/.append style={fill=brown!10}]
    \node[macromolecule subunit] (subigf) {IGF};
    \node[macromolecule subunit, below = 0.1cm of subigf] (subigfr) {IGFR};
    \node[complex, subunits = (subigf)(subigfr)] (comp) {};

    \node[association, connectors = horizontal, left = of comp.200] (p1) {};

    \node[macromolecule, left = of p1] (igfr) {IGFR};

    \node[macromolecule, above left = 2cm and 1.5cm of p1] (igf) {IGF};

    \node[generic process, connectors = horizontal, below = 2.5cm of comp] (p2) {};

    \node[macromolecule, left = of p2] (irs1-4) {IRS1-4};
    \node[sv] at (irs1-4.north) {};

    \node[macromolecule, right = of p2] (pirs1-4) {IRS1-4};
    \node[sv] at (pirs1-4.north) {P};

    \node[simple chemical, clone, above left = of p2] (atp1) {ATP};

    \node[simple chemical, clone, above right = of p2] (adp1) {ADP};

    \node[generic process, connectors = horizontal, below = of pirs1-4] (p3) {};

    \node[macromolecule, right = of p3] (inactivegrb2) {Grb2};
    \node[sv] at (inactivegrb2.south) {inactive};

    \node[macromolecule, left = of p3] (activegrb2) {Grb2};
    \node[sv] at (activegrb2.south) {active};

    \node[association, connectors = vertical, below left = of activegrb2] (p4) {};

    \node[macromolecule, above left = of p4] (sos) {SOS};

    \node[macromolecule subunit, below right = 1.7cm and 0.05cm of p4.south] (sub-activegrb2) {Grb2};
    \node[sv] at (sub-activegrb2.south) {active};
    \node[macromolecule subunit, below left = 1.7cm and 0.05cm of p4.south] (sub-sos) {SOS};
    \node[complex, subunits = (sub-sos)(sub-activegrb2)] (comp2) {};

    \node[generic process, connectors = horizontal, below = of comp2] (p5) {};

    \node[macromolecule, left = of p5] (inactiveras) {RAS};
    \node[sv] at (inactiveras.south) {inactive};

    \node[macromolecule, right = of p5] (activeras) {RAS};
    \node[sv] at (activeras.south) {active};

    \node[simple chemical, below left = of p5] (gtp) {GTP};

    \node[simple chemical, below right = of p5] (gdp) {GDP};

    \node[submap, right = 0.75cm of activeras, align = center, minimum width = 120pt] (submap) {MAPK\\cascade};
    \node[tag, signal to = east, anchor = west] at (submap.west) (tagras) {RAS};
    \node[tag, signal to = west, anchor = east] at (submap.east) (tagerk) {ERK};

    \node[macromolecule, right = 0.75cm of submap] (2perk) {ERK};
    \node[sv] at (2perk.220) {2P};

    \node[generic process, connectors = horizontal, above right = 2.5cm and 3cm of 2perk] (p6) {};

    \node[macromolecule, left = of p6] (rsk) {RSK};
    \node[sv] at (rsk.north) {};

    \node[macromolecule, right = 3.3cm of p6] (prsk) {RSK};
    \node[sv] at (prsk.north) {P};

    \node[simple chemical, clone, above left = of p6] (atp2) {ATP};

    \node[simple chemical, clone, above right = of p6] (adp2) {ADP};

    \node[generic process, connectors = horizontal, below = 2.5cm of prsk] (p7) {};

    \node[macromolecule, left = of p7] (c-fos) {c-Fos};
    \node[sv] at (c-fos.north) {};

    \node[macromolecule, right = of p7] (pc-fos) {c-Fos};
    \node[sv] at (pc-fos.north) {P};

    \node[simple chemical, clone, above left = of p7] (atp3) {ATP};

    \node[simple chemical, clone, above right = of p7] (adp3) {ADP};

    \node[phenotype, below = of pc-fos, align = center] (pheno) {gene\\transcription};

    \draw[consumption] (igf) -- (p1.west);
    \draw[consumption] (igfr) -- (p1.west);
    \draw[production] (p1.east) -- (comp.200);
    \draw[catalysis] (comp) -- (p2);
    \draw[consumption] (atp1) -- (p2.west);
    \draw[consumption] (irs1-4) -- (p2.west);
    \draw[production] (p2.east) -- (adp1);
    \draw[production] (p2.east) -- (pirs1-4);
    \draw[stimulation] (pirs1-4) -- (p3);
    \draw[consumption] (inactivegrb2) -- (p3.east);
    \draw[production] (p3.west) -- (activegrb2);
    \draw[consumption] (activegrb2) -- (p4.north);
    \draw[consumption] (sos) -- (p4.north);
    \draw[production] (p4.south) -- (comp2);
    \draw[stimulation] (comp2) -- (p5);
    \draw[consumption] (inactiveras) -- (p5.west);
    \draw[consumption] (gtp) -- (p5.west);
    \draw[production] (p5.east) -- (activeras);
    \draw[production] (p5.east) -- (gdp);
    \draw[equivalence arc] (activeras) -- (tagras);
    \draw[equivalence arc] (tagerk) -- (2perk);
    \draw[catalysis] (2perk) -| (p6);
    \draw[consumption] (rsk) -- (p6.west);
    \draw[consumption] (atp2) -- (p6.west);
    \draw[production] (p6.east) -- (prsk);
    \draw[production] (p6.east) -- (adp2);
    \draw[catalysis] (prsk) -- (p7);
    \draw[consumption] (c-fos) -- (p7.west);
    \draw[consumption] (atp3) -- (p7.west);
    \draw[production] (p7.east) -- (pc-fos);
    \draw[production] (p7.east) -- (adp3);
    \draw[stimulation] (pc-fos) -- (pheno);
    \draw[compartment, rounded corners=10pt] let \p1=(comp.160), \p2=(igfr.west), \p3=(pheno.south east), \p4=(comp.20) in
    (\p1) -- ($(\x2,\y1)+(-1cm,0)$)
    -- ($(\x2,\y3)+(-1cm,-1cm)$)
    -- ($(\x3,\y3)+(1cm,-1cm)$)
    -- ($(\x3,\y4)+(1cm,0)$)
    -- (\p4)
    node[below, pos = 0.8] {\huge cytosol}
    -- cycle
    ;
    \draw[compartment, rounded corners=10pt] let \p1=(comp.150), \p2=(igfr.west), \p3=(pheno.south east), \p4=(comp.30), \p5=(igf.north) in
    (\p1) -- ($(\x2,\y1)+(-1cm,0)$)
    -- ($(\x2,\y5)+(-1cm,1cm)$)
    -- ($(\x3,\y5)+(1cm,1cm)$)
    node[below, pos = 0.46] {\huge extracelullar}
    -- ($(\x3,\y4)+(1cm,0)$)
    -- (\p4)
    ;
\end{tikzpicture}
}
\end{center}

\newpage

\tableofcontents

\section{Introduction}

\subsection{About}

\emph{sbgntikz} is a \texttikz~\cite{tikz} library to draw SBGN PD, AF or ER maps~\cite{sbgn} directly into \LaTeX{} documents.
It basically encodes SBGN glyphs into \texttikz{} shapes and arrowheads named by keywords, making them straightforwardly drawable within a \texttikz{} picture.
Drawing a specific glyph is then as simple as specifying its corresponding keyword in some \texttikz{} command.

The present manual is intended for an audience that knows SBGN but not particularly \texttikz{}.
The rest of the present section is dedicated to the first steps in using \emph{sbgntikz}: installing the library and drawing a first map (while introducing some basic \texttikz{} syntax).
Section~\ref{sec:draw} references all glyphs and their associated keywords, whereas section~\ref{sec:eff} gives some \texttikz{} options and syntaxes that I find most useful to draw SBGN maps.
I believe users already familiar with \texttikz{} will mostly be interested in reading section~\ref{sec:draw}, and might have different (and maybe better) solutions to the issues presented in section~\ref{sec:eff}.


\subsection{Installation and usage}

The directory \mintinline{bash}{tikz-sbgn/} should be copied to a directory where it can be found by the \TeX{} engine:
\begin{itemize}
    \item in the directory of your \LaTeX{} source file
    \item in your local texmf directory (\mintinline{bash}{/home/<user>/texmf/} under Linux,
        \\\mintinline{bash}{/Users/Library/texmf/} under MacOS).
\end{itemize}

Usually, \texttikz{} is installed within your \TeX{} distribution, so \texttikz{} and \emph{sbgntikz} can be imported directly into your \LaTeX{} source file with no further installation adding the following two commands to your preamble:

\begin{minted}[bgcolor=gray!10]{latex}
\usepackage{tikz}
\usetikzlibrary{sbgn}
\end{minted}

An SBGN map can then be drawn within a \texttikz{} picture using the \colkey{sbgn} key:

\begin{minted}{latex}
\begin{tikzpicture}[sbgn]
% tikz code to draw an SBGN map
\end{tikzpicture}
\end{minted}

\subsection{A first map}

SBGN is all about drawing nodes with specific shapes and arcs with specific arrow heads.
Fortunately, drawing \texttikz{} pictures is not different, making it pretty straightforward to draw SBGN maps using \emph{sbgntikz}: the \mintinline{tex}{\node} command is used to draw nodes, while the \mintinline{tex}{\draw} command is used to draw arcs.
The code to draw an SBGN node (or an attribute) will usually look like the following:

\begin{minted}[bgcolor=gray!10]{latex}
\node[<sbgn node>, ...] (name) at (point) {LABEL};
\end{minted}

where
\begin{itemize}
\item \mintinline{latex}{<sbgn node>} is a keyword corresponding to the type of node to be drawn (e.g. \mintinline{latex}{simple chemical} for a simple chemical);
\item \mintinline{latex}{...} is a list of other options for the node (e.g. its relative positioning towards another node, color, line width \dots);
\item \mintinline{latex}{(name)} specifies the name of the node (optional);
\item \mintinline{latex}{at (point)} specifies the point on the canvas where to draw the node (optional, by default \mintinline{latex}{(0,0)} if no relative positioning is specified in the nodes' options);
\item \mintinline{latex}{{LABEL}} specifies the label of the node that will be displayed (mandatory but can be empty).
\end{itemize}

As for arcs, they can be drawn using the following piece of code:

\begin{minted}[bgcolor=gray!10]{latex}
\node[<sbgn arc>, ...] (a) -- (b);
\end{minted}

where

\begin{itemize}
\item \mintinline{latex}{<sbgn arc>} is a keyword corresponding to the type of arc to be drawn (e.g. \mintinline{latex}{necessary stimulation} for a necessary stimulation);
\item \mintinline{latex}{...} is a list of other options for the arc (e.g. its color, line width \dots);
\item \mintinline{latex}{(a)} is a point on the canvas or the name of a node from which the arc will depart;
\item \mintinline{latex}{(b)} is a point on the canvas or the name of a node on which the arc will arrive.
\end{itemize}

Knowing those two basic syntaxes, one can draw pretty much any desired SBGN map.
Following is an example of code to draw a simple PD map.
It relies on relative positioning provided by \texttikz{}'s \mintinline{tex}{positioning} library, as positioning all nodes with absolute coordinates would be too cumbersome (see section~\ref{sec:eff} for few more details, or the \tikzmanual{} for lot more details).


\begin{center}
\scriptsize
\begin{minted}[bgcolor=gray!10, breaklines = true]{latex}
\documentclass{standalone}

\usepackage{tikz}
\usetikzlibrary{positioning, sbgn}

\begin{document}

\begin{tikzpicture}[sbgn]
\node[macromolecule] (erk) {ERK};   % this node has no absolute nor relative positioning, so it is placed at (0,0) by default
\node[sv] at (erk.120) {};  % the state variable is placed on the border of the node, at an angle of 120 deg
\node[generic process, connectors = horizontal, right = of erk] (p) {}; % we add connectors, and use relative positioning
\node[macromolecule, right = of p] (perk) {ERK};
\node[sv] at (perk.120) {P};
\node[simple chemical, below left = of p] (atp) {ATP};
\node[simple chemical, below right = of p] (adp) {ADP};
\node[macromolecule, above = 2cm of p] (pmek) {MEK};
\node[sv] at (pmek.120) {P};

\draw[consumption] (erk) -- (p.west);   % p being the name of the process node, p.west is the tip of its left connector
\draw[consumption] (atp) to [bend left=40] (p.west);    % arcs can be bent using a specific syntax, where "--" is replaced by "to [bend <direction>=<angle>]"
\draw[production] (p.east) -- (perk);
\draw[production] (p.east) to [bend left=40] (adp);
\draw[catalysis] (pmek) -- (p);
\end{tikzpicture}

\end{document}
\end{minted}
\normalsize
\end{center}

Compiling the above code would produce the following figure:

\begin{center}
\colorbox{blue!10}{
\begin{tikzpicture}[sbgn]
    %ERK
    \node[macromolecule] (erk) {ERK};
    \node[sv] (sv-erk) at (erk.120) {};
    %process
    \node[generic process, connectors = horizontal, right = of erk] (p) {};
    %p-ERK
    \node[macromolecule, right = of p] (perk) {ERK};
    \node[sv] (sv-erk) at (perk.120) {P};
    %atp
    \node[simple chemical, below left = of p] (atp) {ATP};
    %adp
    \node[simple chemical, below right = of p] (adp) {ADP};
    %p-MEK
    \node[macromolecule, above = 2cm of p] (pmek) {MEK};
    \node[sv] at (pmek.120) {P};
    %arcs
    \draw[consumption] (erk) -- (p.west);
    \draw[consumption] (atp) to [bend left=40] (p.west);
    \draw[production] (p.east) -- (perk);
    \draw[production] (p.east) to [bend left=40] (adp);
    \draw[catalysis] (pmek) -- (p);
\end{tikzpicture}
}
\end{center}

\section{Drawing glyphs}
\label{sec:draw}
\subsection{Nodes}

\begin{entry}{Macromolecule}{macromolecule}{PD}
\begin{code}
\node[macromolecule] {LABEL};
\end{code}
\end{entry}

\begin{entry}{Macromolecule multimer}{macromolecule multimer}{PD}
\begin{code}
\node[macromolecule multimer] {LABEL};
\end{code}
\end{entry}

\begin{entry}{Nucleic acid feature}{nucleic acid feature}{PD}
\begin{code}
\node[nucleic acid feature] {LABEL};
\end{code}
\end{entry}

\begin{entry}{Nucleic acid feature multimer}{nucleic acid feature multimer}{PD}
\begin{code}
\node[nucleic acid feature multimer] {LABEL};
\end{code}
\end{entry}

\begin{entry}{Unspecified entity}{unspecified entity}{PD}
\begin{code}
\node[unspecified entity] {LABEL};
\end{code}
\end{entry}

\begin{entry}{Simple chemical}{simple chemical}{PD}
\begin{code}
\node[simple chemical] {LABEL};
\end{code}
\end{entry}

\begin{entry}{Simple chemical multimer}{simple chemical multimer}{PD}
\begin{code}
\node[simple chemical multimer] {LABEL};
\end{code}
\end{entry}

\begin{entry}{Complex}{complex}{PD}
\begin{code}
\node[complex] {LABEL};
\end{code}
\end{entry}

\begin{entry}{Complex multimer}{complex multimer}{PD}
\begin{code}
\node[complex multimer] {LABEL};
\end{code}
\end{entry}

\begin{entry}{Perturbation}{perturbation}{PD}
\begin{code}
\node[perturbation] {LABEL};
\end{code}
\end{entry}

\begin{entry}{Emptyset}{emptyset}{PD}
\begin{code}
\node[emptyset] {};
\end{code}
\end{entry}

\begin{entry}{Biological activity}{biological activity}{AF}
\begin{code}
\node[biological activity] {LABEL};
\end{code}
\end{entry}

\begin{entry}{Entity}{entity}{ER}
\begin{code}
\node[entity] {LABEL};
\end{code}
\end{entry}

\begin{entry}{Outcome}{outcome}{ER}
\begin{code}
\node[outcome] {};
\end{code}
\end{entry}

\begin{entry}{Value}{value}{ER}
\begin{code}
\node[value] {val};
\end{code}
\end{entry}

\begin{entry}{Compartment}{compartment}{PD, AF, ER}
\begin{code}
\node[compartment, shape=rectangle, minimum width=85pt, minimum height=70pt, rounded corners=10pt] {LABEL};
\end{code}
\end{entry}

\begin{entry}{Connectors}{connectors=\{vertical, horizontal\}}{PD}
Vertical or horizontal connectors can be added to process nodes and logical operators.

\begin{code}
\node[generic process, connectors=horizontal] {};
\end{code}

\begin{code}
\node[generic process, connectors=vertical] {};
\end{code}

Cardinal anchors are set to the tip of the connectors.
Other border anchors are not changed.

\begin{code}
\node[and, connectors=horizontal] (o) {};
\path[draw=blue] (o.west) circle[radius=2pt];
\path[draw=red] (o.180) circle[radius=2pt];
\end{code}
\end{entry}

\begin{entry}{Clone}{clone}{PD}
\begin{code}
\node[unspecified entity, clone] {LABEL};
\end{code}
\end{entry}

\begin{entry}{Generic process}{generic process}{PD}
\begin{code}
\node[generic process] {};
\end{code}
\end{entry}

\begin{entry}{Omitted process}{omitted process}{PD}
\begin{code}
\node[omitted process] {};
\end{code}
\end{entry}

\begin{entry}{Unknown process}{unknown process}{PD}
\begin{code}
\node[unknown process] {};
\end{code}
\end{entry}

\begin{entry}{Association}{association}{PD}
\begin{code}
\node[association] {};
\end{code}
\end{entry}

\begin{entry}{Dissociation}{dissociation}{PD}
\begin{code}
\node[dissociation] {};
\end{code}
\end{entry}

\begin{entry}{Phenotype}{phenotype}{PD, AF, ER}
\begin{code}
\node[phenotype] {LABEL};
\end{code}
\end{entry}

\begin{entry}{And}{and}{PD, AF, ER}
\begin{code}
\node[and] {};
\end{code}
\end{entry}

\begin{entry}{Or}{or}{PD, AF, ER}
\begin{code}
\node[or] {};
\end{code}
\end{entry}

\begin{entry}{Not}{not}{PD, AF, ER}
\begin{code}
\node[not] {};
\end{code}
\end{entry}

\begin{entry}{Delay}{delay}{PD, AF, ER}
\begin{code}
\node[delay] {};
\end{code}
\end{entry}

\begin{entry}{Submap}{submap}{PD}
\begin{code}
\node[submap, minimum width=85pt, minimum height=70pt] {LABEL};
\end{code}
\end{entry}

\begin{entry}{Tag}{tag}{PD}
\begin{code}
\node[tag] {LABEL};
\end{code}
\end{entry}

\subsection{Arcs}

\begin{entry}{Logic arc}{logic arc}{PD, AF, ER}
\begin{code}
\draw[logic arc] (0,0) -- (2cm,0);
\end{code}
\end{entry}

\begin{entry}{Equivalence arc}{equivalence arc}{PD}
\begin{code}
\draw[equivalence arc] (0,0) -- (2cm,0);
\end{code}
\end{entry}

\begin{entry}{Consumption}{consumption}{PD}
\begin{code}
\draw[consumption] (0,0) -- (2cm,0);
\end{code}
\end{entry}

\begin{entry}{Production}{production}{PD}
\begin{code}
\draw[production] (0,0) -- (2cm,0);
\end{code}
\end{entry}

\begin{entry}{Reversible}{reversible}{PD}
\begin{code}
\draw[reversible] (0,0) -- (2cm,0);
\end{code}
\end{entry}

\begin{entry}{Modulation}{modulation}{PD, AF, ER}
\begin{code}
\draw[modulation] (0,0) -- (2cm,0);
\end{code}
\end{entry}

\begin{entry}{Stimulation}{stimulation}{PD, AF, ER}
\begin{code}
\draw[stimulation] (0,0) -- (2cm,0);
\end{code}
\end{entry}

\begin{entry}{Necessary stimulation}{necessary stimulation}{PD, AF, ER}
\begin{code}
\draw[necessary stimulation] (0,0) -- (2cm,0);
\end{code}
\end{entry}

\begin{entry}{Absolute stimulation}{absolute stimulation}{ER}
\begin{code}
\draw[absolute stimulation] (0,0) -- (2cm,0);
\end{code}
\end{entry}

\begin{entry}{Inhibition}{inhibition}{PD, AF, ER}
\begin{code}
\draw[inhibition] (0,0) -- (2cm,0);
\end{code}
\end{entry}

\begin{entry}{Absolute inhibition}{absolute inhibition}{ER}
\begin{code}
\draw[absolute inhibition] (0,0) -- (2cm,0);
\end{code}
\end{entry}

\begin{entry}{Assignment}{assignment}{ER}
\begin{code}
\draw[assignment] (0,0) -- (2cm,0);
\end{code}
\end{entry}

\begin{entry}{Interaction}{interaction}{ER}
\begin{code}
\draw[interaction] (0,0) -- (2cm,0);
\end{code}

N-ary interactions can be drawn using the \colkey{nary} node:

\begin{code}
\draw[interaction] (0,0) -- node[nary, pos=0.5] (a) {} (2cm,0);
\node[outcome] at (a.100) {};
\end{code}
\end{entry}

\begin{entry}{Anchor point}{anchor point}{ER}
\begin{code}
\draw[interaction] (0,0) -- coordinate[anchor point, pos = 0.5] (a) (2,0);
\draw[stimulation] (1,1) -- (a);
\end{code}
\end{entry}

\subsection{Nodes' and arcs' attributes}

\begin{entry}{State variable}{sv}{PD, ER}
\begin{code}
\node[entity, draw=gray!60] (m) {};
\node[sv] at (m.north) {val@var};
\end{code}
\end{entry}

\begin{entry}{Existence state variable}{sv existence}{ER}
\begin{code}
\node[entity, draw=gray!60] (m) {};
\node[sv existence] at (m.north) {};
\end{code}
\end{entry}

\begin{entry}{Location state variable}{sv location}{ER}
\begin{code}
\node[entity, draw=gray!60] (m) {};
\node[sv location] at (m.north) {};
\end{code}
\end{entry}

\begin{entry}{Unit of information}{ui}{PD, ER}
\begin{code}
\node[entity, draw=gray!60] (m) {};
\node[ui] at (m.north) {pre:label};
\end{code}

In PD, stoichiometry can be drawn using a unit of information along an arc:

\begin{code}
\draw[production] (0,0) -- node[ui, above, pos=0.5] {N:5} (2cm,0);
\end{code}
\end{entry}

\begin{entry}{Unit of information simple chemical}{ui simple chemical}{AF}
\begin{code}
\node[biological activity, draw=gray!60] (m) {};
\node[ui simple chemical] at (m.north) {LABEL};
\end{code}
\end{entry}

\begin{entry}{Unit of information nucleic acid feature}{ui nucleic acid feature}{AF}
\begin{code}
\node[biological activity, draw=gray!60] (m) {};
\node[ui nucleic acid feature] at (m.north) {LABEL};
\end{code}
\end{entry}

\begin{entry}{Unit of information macromolecule}{ui macromolecule}{AF}
\begin{code}
\node[biological activity, draw=gray!60] (m) {};
\node[ui macromolecule] at (m.north) {LABEL};
\end{code}
\end{entry}

\begin{entry}{Unit of information perturbation}{ui perturbation}{AF}
\begin{code}
\node[biological activity, draw=gray!60] (m) {};
\node[ui perturbation] at (m.north) {LABEL};
\end{code}
\end{entry}

\begin{entry}{Unit of information complex}{ui complex}{AF}
\begin{code}
\node[biological activity, draw=gray!60] (m) {};
\node[ui complex] at (m.north) {LABEL};
\end{code}
\end{entry}

\begin{entry}{Unspecidied entity subunit}{unspecified entity subunit}{PD}
\begin{code}
\node[unspecified entity subunit] {LABEL};
\end{code}
\end{entry}

\begin{entry}{Macromolecule subunit}{macromolecule subunit}{PD}
\begin{code}
\node[macromolecule subunit] {LABEL};
\end{code}
\end{entry}

\begin{entry}{Macromolecule multimer subunit}{macromolecule multimer subunit}{PD}
\begin{code}
\node[macromolecule multimer subunit] {LABEL};
\end{code}
\end{entry}

\begin{entry}{Nucleic acid feature subunit}{nucleic acid feature subunit}{PD}
\begin{code}
\node[nucleic acid feature subunit] {LABEL};
\end{code}
\end{entry}

\begin{entry}{Nucleic acid feature multimer subunit}{nucleic acid feature multimer subunit}{PD}
\begin{code}
\node[nucleic acid feature multimer subunit] {LABEL};
\end{code}
\end{entry}

\begin{entry}{Simple chemical subunit}{simple chemical subunit}{PD}
\begin{code}
\node[simple chemical subunit] {LABEL};
\end{code}
\end{entry}

\begin{entry}{Simple chemical multimer}{simple chemical multimer subunit}{PD}
\begin{code}
\node[simple chemical multimer subunit] {LABEL};
\end{code}
\end{entry}

\begin{entry}{Complex subunit}{complex subunit}{PD}
\begin{code}
\node[complex subunit] {LABEL};
\end{code}
\end{entry}

\begin{entry}{Complex multimer subunit}{complex multimer subunit}{PD}
\begin{code}
\node[complex multimer subunit] {LABEL};
\end{code}
\end{entry}

\begin{entry}{Subunits}{subunits}{PD}

The \colkey{subunits} key allows drawing complexes around subunits:

\begin{code}
\node[unspecified entity subunit] (m1) {A};
\node[unspecified entity subunit, below=0.1cm of m1] (m2) {B};
\node[complex, subunits=(m1)(m2)] {};
\end{code}
\end{entry}

\section{Customizing and drawing maps effectively}
\label{sec:eff}
\subsection{Useful options for nodes and arcs}

The style of nodes and arcs (and their attributes) can be customized at will using the numerous options offered by \texttikz{}.
Following are a few options that might be useful for customizing SBGN maps.\\

\begin{entry}{Foreground color}{draw}{Nodes, Arcs}
\begin{code}
\node[generic process, draw = red] (p) {};
\node[unspecified entity, draw = blue, below = of p] (m) {};
\draw[production, draw = green] (p) -- (m);
\end{code}
\end{entry}

\begin{entry}{Background color}{fill}{Nodes}
\begin{code}
\node[unspecified entity, fill = green!120] (m) {};
\end{code}
\end{entry}

\begin{entry}{Line width}{line width}{Nodes, Arcs}
\begin{code}
\node[unspecified entity, line width = 3pt] (m) {};
\draw[absolute stimulation, line width = 0.2pt] (m) -- (0cm,-2cm);
\end{code}
\end{entry}

\begin{entry}{Minimum width}{minimum width}{Nodes}
\begin{code}
\node[complex, minimum width = 20pt] (m) {};
\node[simple chemical, minimum width = 70pt] at (0cm,-2cm) (m) {};
\end{code}
\end{entry}

\begin{entry}{Minimum height}{minimum width}{Nodes}
\begin{code}
\node[complex, minimum height = 20pt] (m) {};
\node[simple chemical, minimum height = 70pt] at (0cm,-2cm) (m) {};
\end{code}
\end{entry}

\subsection{Positioning of nodes, arcs, and their attributes}

\subsection{Bended arcs, multi-part arcs}
\index{bended arc}

\texttikz{} offers a simple way to bend arcs with the following syntax:
\begin{minted}[bgcolor=gray!10]{latex}
\draw (a) to [in=<in_angle>, out=<out_angle>] (b);
\end{minted}

where \mintinline{latex}{<in_angle>} specifies the angle at which the arc leaves the source point or node and \mintinline{latex}{<out_angle>} the angle at which the arc arrives on the target point or node.
Both angles are defined relatively to the picture's coordinate.

One can also use the following shortcut:

\begin{minted}[bgcolor=gray!10]{latex}
\draw (a) to [bend <direction>=<angle>] (b);
\end{minted}

where \mintinline{latex}{<direction>={left, right}} specifies the direction where to bend the arc and \mintinline{latex}{<angle>} the angle at which the arc leaves the source point or node.
The angle is this time defined relatively to the line passing through both points/nodes.

\begin{code}
\node[biological activity] (a) {A};
\node[biological activity, below = of a] (b) {B};
\draw[modulation] (a) to [out=-120, in=80] (b);
\draw[stimulation] (a) to [bend left=40] (b);
\draw[inhibition] (a) to [bend right=80] (b);
\end{code}

\index{multi-part arc}

It is often necessary to break arcs into muliple parts for improved readability.
\texttikz{} offers very simple operations to break arcs into horizontal and vertical sub-parts, that replace the default \mintinline{latex}{--} operation.
The \mintinline{latex}{|-} operation will produce an horizontal sub-part followed by a vertical one, and the \mintinline{latex}{-|} a vertical sub-part followed by a horizontal one.
It can also be convenient to use the \mintinline{latex}{--+} and \mintinline{latex}{--++} to draw arcs with more than two sub-parts.

\begin{code}
\node[biological activity] (a) {A};
\node[biological activity, below right = 1.5cm and 0.3 cm of a.center] (b) {B};
\draw[stimulation] (a.240) |- (b);
\draw[inhibition] (a) -| (b);
\draw[modulation] (b.240) --++ (0,-1) --++ (1,0) --++ (0,1);
\end{code}


\subsection{Nodes along paths for ER maps}

Drawing SBGN ER maps is particular considering that one might have to draw arcs targeting other arcs.
This is not straightforwardly possible in \texttikz{}, as arcs must target points or nodes.

\section{Examples}
\section{License}

\bibliographystyle{unsrt}
\bibliography{sbgntikz_v0_1}

\printindex

\end{document}
