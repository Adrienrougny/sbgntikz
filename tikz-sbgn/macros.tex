%Copyright 2018 by Adrien Rougny
%This file may be distributed and/or modified under the terms of the GNU General Public License version 3
%See the file LICENSE.txt
\def\true{true}
\def\false{false}
\def\vertical{vertical}
\def\horizontal{horizontal}

\def\pointborderrectanglecorners#1#2#3{%point, southwest, northeast
    \pgf@process{#2}
    \pgf@xa = \pgf@x
    \pgf@ya = \pgf@y
    \pgf@process{#3}
    \pgf@xb = \pgf@x
    \pgf@yb = \pgf@y
    \pgf@xc = 0.5\pgf@xa
    \advance\pgf@xc by 0.5\pgf@xb
    \pgf@yc = 0.5\pgf@ya
    \advance\pgf@yc by 0.5\pgf@yb
    \advance\pgf@xb by -\pgf@xc
    \advance\pgf@yb by -\pgf@yc
    \pgf@process{#1}
    \pgf@xa = \pgf@x
    \pgf@ya = \pgf@y
    \advance\pgf@xa by -\pgf@xc
    \advance\pgf@ya by -\pgf@yc
    \edef\pgf@marshal{
        \noexpand\pgfpointborderrectangle{\noexpand\pgfpoint{\the\pgf@xa}{\the\pgf@ya}}{\noexpand\pgfpoint{\the\pgf@xb}{\the\pgf@yb}}
    }
    \pgf@process{\pgf@marshal}
    \advance\pgf@x by \pgf@xc
    \advance\pgf@y by \pgf@yc
}

\def\pointbordercircle#1#2#3{%point, center, radius
    \pgf@process{#2}
    \pgf@xc = \pgf@x
    \pgf@yc = \pgf@y
    \pgf@process{#1}
    \pgf@xa = \pgf@x
    \pgf@ya = \pgf@y
    \pgf@xb = #3
    \pgf@yb = #3
    \edef\pgf@marshal{%
        \noexpand\pgfpointborderellipse
        {\noexpand\pgfpoint{\the\pgf@xa}{\the\pgf@ya}}
        {\noexpand\pgfpoint{\the\pgf@xb}{\the\pgf@yb}}%
    }
    \pgf@process{\pgf@marshal}
    \advance\pgf@x by \pgf@xc
    \advance\pgf@y by \pgf@yc
}

\def\circlefindx#1#2#3#4{%center, radius, y, sol
    \pgf@process{#1}
    \@tempdima=\pgf@x
    \@tempdimb=\pgf@y
    \pgf@xb=#2
    \pgf@y=#3
    \advance\pgf@y by -\@tempdimb
    \pgfmathsetlength{\pgf@x}{sqrt(\pgf@xb^2-\pgf@y^2)}
    \ifnum #4=1
        \pgf@x=-\pgf@x
    \fi
    \advance\pgf@x by \@tempdima
}

\def\circlefindangle#1#2#3#4{%center, radius, y, sol
    \pgf@process{#1}
    \@tempdima=\pgf@x
    \@tempdimb=\pgf@y
    \pgf@xb=#2
    \pgf@y=#3
    \advance\pgf@y by -\@tempdimb
    \pgfmathsetlength{\pgf@x}{sqrt(\pgf@xb^2-\pgf@y^2)}
    \ifnum #4=1
        \pgf@x=-\pgf@x
    \fi
    % \pgfmathsetlength{\pgf@x}{atan2(\pgf@y,\pgf@x)}
    \pgfmathsetlength{\pgf@x}{round(atan2(\pgf@y,\pgf@x))}
}

\def\belongstocone#1#2#3#4{%center, point1, point2, point
    % \global\if@tempswa
    \pgf@process{#1}
    \pgf@xc=\pgf@x
    \pgf@yc=\pgf@y
    \pgf@process{#2}
    \advance\pgf@x by -\pgf@xc
    \advance\pgf@y by -\pgf@yc
    \ifdim\pgf@y=0pt
        \ifdim\pgf@x>0pt
            \pgf@xa=0pt
        \else
            \pgf@xa=180pt
        \fi
    \else
        \pgfmathsetlength{\pgf@xa}{atan2(\pgf@y,\pgf@x)}
    \fi
    % \ifdim\pgf@xa<0pt
    %     \advance\pgf@xa by 360pt
    % \fi
    \pgf@process{#3}
    \advance\pgf@x by -\pgf@xc
    \advance\pgf@y by -\pgf@yc
    \ifdim\pgf@y=0pt
        \ifdim\pgf@x>0pt
            \pgf@xb=0pt
        \else
            \pgf@xb=180pt
        \fi
    \else
        \pgfmathsetlength{\pgf@xb}{atan2(\pgf@y,\pgf@x)}
    \fi
    % \ifdim\pgf@xb<0pt
    %     \advance\pgf@xb by 360pt
    % \fi
    \pgf@process{#4}
    \advance\pgf@x by -\pgf@xc
    \advance\pgf@y by -\pgf@yc
    \ifdim\pgf@y=0pt
        \ifdim\pgf@x>0pt
            \pgf@ya=0pt
        \else
            \pgf@ya=180pt
        \fi
    \else
        \pgfmathsetlength{\pgf@ya}{atan2(\pgf@y,\pgf@x)}
    \fi
    % \ifdim\pgf@ya<0pt
    %     \advance\pgf@ya by 360pt
    % \fi
    \ifdim\pgf@xa<0pt
        \advance\pgf@xa by 360pt
        \ifdim\pgf@xb<0pt
            \advance\pgf@xb by 360pt
        \fi
        \ifdim\pgf@ya<0pt
            \advance\pgf@ya by 360pt
        \fi
    \fi
    \ifdim\pgf@ya>\pgf@xa
        \global\@tempswafalse
    \else
        \ifdim\pgf@ya<\pgf@xb
            \global\@tempswafalse
        \else
            \global\@tempswatrue
        \fi
    \fi
}

\def\intersectionlinecircle#1#2#3#4#5{% point1, point2, center, radius, solution
    \pgf@process{#1}
    \pgf@xa = \pgf@x
    \pgf@ya = \pgf@y
    \pgf@process{#2}
    \pgf@xb = \pgf@x
    \pgf@yb = \pgf@y
    \pgf@process{#3}
    \pgf@xc = \pgf@x
    \pgf@yc = \pgf@y
    \newdimen\rad
    \rad = #4
    \c@pgf@counta=#5
    \advance\pgf@xa by -\pgf@xc
    \advance\pgf@xb by -\pgf@xc
    \advance\pgf@ya by -\pgf@yc
    \advance\pgf@yb by -\pgf@yc
    \ifdim\pgf@xa=\pgf@xb % infinite slope set to max dimen
        \message{the two are equal}
        \ifnum\c@pgf@counta=1
            \pgfmathsetlength{\pgf@y}{-sqrt(\rad^2-\pgf@xa^2)}
        \else
            \pgfmathsetlength{\pgf@y}{sqrt(\rad^2-\pgf@xa^2)}
        \fi
        \pgf@x=\pgf@xa
    \else
        \pgfmathsetlength{\@tempdima}{(\pgf@yb-\pgf@ya)/(\pgf@xb-\pgf@xa)}
        \pgfmathsetlength{\@tempdimb}{\pgf@yb-\pgf@xb*\@tempdima}
        \pgfkeys{/pgf/fpu} % for bigger dimen
        \ifnum\c@pgf@counta=1

            % \pgfmathsetlength{\pgf@x}{(-sqrt(\rad^2*\@tempdima^2+\rad^2-\@tempdimb^2) - \@tempdimb*\@tempdima)/(\@tempdima^2+1)}
            \pgfmathparse{(-sqrt(\rad^2*\@tempdima^2+\rad^2-\@tempdimb^2) - \@tempdimb*\@tempdima)/(\@tempdima^2+1)}
            \pgfmathfloattofixed{\pgfmathresult}
            \pgf@x=\pgfmathresult pt
        \else
            % \pgftext{\the\@tempdimb}
            \pgfmathparse{(sqrt(\rad^2*\@tempdima^2+\rad^2-\@tempdimb^2) - \@tempdimb*\@tempdima)/(\@tempdima^2+1)}
            \pgfmathfloattofixed{\pgfmathresult}
            \pgf@x=\pgfmathresult pt

        \fi
        \pgfkeys{/pgf/fpu=false} % for bigger dimen
        \pgfmathsetlength{\pgf@y}{\@tempdima*\pgf@x+\@tempdimb}
    \fi
    \advance\pgf@x by \pgf@xc
    \advance\pgf@y by \pgf@yc
}

\def\sbgn@subunits#1{
    \pgfkeysalso{/tikz/fit=#1}
    \pgfkeysalso{
        /tikz/text width/.expand once=\tikz@text@width+10pt,
        /tikz/text height/.expand once=\tikz@text@height+8pt,
        /tikz/text depth/.expand once=\tikz@text@depth+10pt
    }
}

\pgfkeys{/tikz/isnode/.initial=false, /tikz/every node/.style={/tikz/isnode=true}}
\pgfkeys{/pgf/.cd, clone/.initial/.expanded=false, clone/.default/.expanded=\true}
\pgfkeys{/pgf/.cd, connectors/.initial=none}
\pgfkeys{/tikz/subunits/.code=\sbgn@subunits{#1}}
% this awesome piece of code was written by Loop Space and taken from https://tex.stackexchange.com/questions/20425/z-level-in-tikz/20426#20426
\pgfkeys{/tikz/draw layer/.code={
    \pgfonlayer{#1}\begingroup
    \aftergroup\endpgfonlayer
    \aftergroup\endgroup
    }
}

\pgfkeys{/tikz/node layer/.code={%
    \gdef\node@@layer{%
      \setbox\tikz@tempbox=\hbox\bgroup\pgfonlayer{#1}\unhbox\tikz@tempbox\endpgfonlayer\egroup}
    \aftergroup\node@layer
  }
}

\def\node@layer{\aftergroup\node@@layer}
