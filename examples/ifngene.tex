%sbgntikz example file
%This SBGN PD map represents the regulation of IRF1's expression.
%It is a reproduction in sbgntikz of the map depicted in the SBGN PD specification [The SBGN PD language Level 1 Version 1.3 – Available at Journal of Integrative Bioinformatics, 2015 Sep 4;12(2):263 (doi: 10.2390/biecoll-jib-2015-263)]

\documentclass{standalone}

\usepackage{tikz}
\usetikzlibrary{positioning, calc, sbgn}

\begin{document}

\begin{tikzpicture}[node distance = 1.5cm]
    \node[submacromolecule] (substat1-1) {STAT1$\alpha$};
    \node[ui] (ui1) at (substat1-1.90) {mt:prot};
    \node[sv] (sv1) at (substat1-1.228) {P@Y701};
    \node[sv] (sv2) at (substat1-1.312) {P@Y727};
    \node[complex, subunits = (substat1-1)(ui1)(sv1)(sv2)] (comp1) {};

    \node[association, below right = 1.5cm and 0.5cm of comp1] (p1) {};

    \node[nucleicacidfeature, above right = 1.5cm and 0.5cm of p1] (irf1-gas) {IRF1-GAS};
    \node[ui] (ui7) at (irf1-gas.90) {ct:grr};

    \node[submacromolecule, below = 2cm of p1] (substat1-2) {STAT1$\alpha$};
    \node[ui] (ui2) at (substat1-2.90) {mt:prot};
    \node[sv] (sv3) at (substat1-2.228) {P@Y701};
    \node[sv] (sv4) at (substat1-2.312) {P@Y727};
    \node[subcomplex, subunits = (substat1-2)(ui2)(sv3)(sv4)] (comp2) {};
    \node[subnucleicacidfeature, below = 0.3cm of comp2] (subirf1-gas) {IRF1-GAS};
    \node[ui] (ui3) at (subirf1-gas.90) {ct:grr};
    \node[complex, subunits = (comp2)(subirf1-gas)(ui3)] (comp3) {};

    \node[and, below right = -1cm and 1.5cm of comp3] (and) {};

    \node[nucleicacidfeature, above right = of and] (irf1-gene) {IRF1};
    \node[ui] (ui4) at (irf1-gene.90) {ct:gene};

    \node[genericprocess, below = of and] (p2) {};

    \node[emptyset, left = of p2] (es1) {};

    \node[nucleicacidfeature, right = of p2] (irf1-mrna) {IRF1};
    \node[ui] (ui5) at (irf1-mrna.90) {ct:mRNA};

    \node[genericprocess, below = of irf1-mrna] (p3) {};

    \node[emptyset, left = of p3] (es2) {};

    \node[macromolecule, right = of p3] (irf1-prot) {IRF1};
    \node[ui] (ui6) at (irf1-prot.90) {mt:prot};

    \draw[consumption] (comp1) -- (p1);
    \draw[consumption] (irf1-gas) -- (p1);
    \draw[production] (p1) -- (comp3);
    \draw[logicarc] (comp3) -- (and);
    \draw[logicarc] (irf1-gene) -- (and);
    \draw[necessarystimulation] (and) -- (p2);
    \draw[consumption] (es1) -- (p2);
    \draw[production] (p2) -- (irf1-mrna);
    \draw[necessarystimulation] (irf1-mrna) -- (p3);
    \draw[consumption] (es2) -- (p3);
    \draw[production] (p3) -- (irf1-prot);

\end{tikzpicture}

\end{document}
