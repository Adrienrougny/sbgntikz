%sbgntikz example file
%This SBGN ER map represents the effect of depolarization on intracellular calcium.
%It is a reproduction in sbgntikz of the map depicted in the SBGN ER specification [The SBGN ER language Level 1 Version 2.0 – Available at Journal of Integrative Bioinformatics, 2015 Sep 4;12(2):264 (doi:10.2390/biecoll-jib-2015-264)]

\documentclass{standalone}

\usepackage{tikz}
\usetikzlibrary{sbgn}
\usetikzlibrary{fit, positioning, calc}

\begin{document}

\begin{tikzpicture}[sbgn, node distance = 2cm]

    \node[entity, minimum width = 110pt] (camkii) {CaMKII};
    \node[sv] (t286) at (camkii.130) {P@T286};
    \node[sv] (t306) at (camkii.50) {T306};
    \node[ui] at (camkii.270) {mt:prot};

    \node[entity, below right = 0.5cm and 3cm of camkii] (cam) {CaM};
    \node[ui] at (cam.270) {mt:prot};

    \node[value, above = 1.3cm of t306] (t306-varp) {P};

    \node[entity, above = 1cm of cam] (ca) {Ca$^{2+}$};
    \node[sv existence] (ex) at (ca.130) {};
    \node[ui] at (ca.50) {mt:ion};

    \node[value, above = 1.3cm of ex] (catrue) {T};

    \node[perturbation, minimum width = 40pt, minimum height = 20pt, above right = 0.8cm and -0.3cm of ca] (dv) {dV};

    \node[entity, below left = 0.5cm and 2cm of camkii] (glur) {GluR};
    \node[sv location] (location) at (glur.130) {};
    \node[value] (glurp) at (glur.50) {P};
    \node[ui] at (glur.270) {mt:prot};

    \node[value, above = 1.3cm of glurp] (glurptrue) {T};

    \node[value, above = 1.3cm of location] (psd) {PSD};

    \node[value, above left = 2.3cm and 0.1cm of t286] (t286true) {T};
    \node[value, above right = 2.3cm and 0.1cm of t286] (t286false) {F};
    \coordinate[above = 1.5cm of t286] (truefalse);

    \draw[assignment] (t286true) -- (truefalse)
        coordinate[anchor point, pos = 0.25] (t286true-truefalse-c1)
        node[outcome, pos = 0.5] (t286true-truefalse-o1) {}
        coordinate[anchor point, pos = 0.8] (t286true-truefalse-c2)
        -- (t286)
    ;
    \draw[assignment] (t286false) -- (truefalse)
        coordinate[anchor point, pos = 0.25] (t286false-truefalse-c1)
        -- (t286)
    ;

    \node[entity, right = 1cm of t286false-truefalse-c1] (ppase) {PPase};
    \node[ui] at (ppase.90) {mt:prot};

    \draw[stimulation] (ppase) -- (t286false-truefalse-c1);

    \coordinate (camkii-camkii-trans-dep) at (camkii.10);
    \coordinate (camkii-camkii-trans-arr) at (camkii.350);
    \draw[interaction] (camkii-camkii-trans-dep) -- ($(camkii-camkii-trans-dep)+(1cm,0)$)
        node[outcome, pos = 0.5] (camkii-camkii-trans-o1) {}
        -- ($(camkii-camkii-trans-arr)+(1cm,0)$)
        node[ui, pos = 0.5] {trans}
        -- (camkii-camkii-trans-arr)
        node[outcome, pos = 0.5] (camkii-camkii-trans-o2) {}
    ;

    \coordinate (camkii-cam-dep) at (camkii.335);
    \coordinate (camkii-cam-arr) at (cam.west);
    \draw[interaction] let \p1=(camkii-cam-arr), \p2=(camkii-camkii-trans-o2) in (camkii-cam-dep) |- (camkii-cam-arr)
        node[outcome, pos = 0.4] (camkii-cam-o1) {}
        node[outcome, pos = 0.78] (camkii-cam-o2) {}
        coordinate[anchor point] (camkii-cam-c2) at (\x2,\y1)
        coordinate[anchor point, pos = 0.85] (camkii-cam-c3)
        coordinate[anchor point, pos = 0.90] (camkii-cam-c4)
    ;

    \coordinate (camkii-camkii-cis-dep) at (camkii.205);
    \coordinate (camkii-camkii-cis-arr) at (camkii.315);
    \draw[interaction] let \p1=(camkii-camkii-cis-dep), \p2=(camkii-camkii-cis-arr), \p3=(camkii-cam-arr), \p4=(camkii-cam-o1) in (\x1,\y1) -- (\x1,\y3)
        node[outcome, pos = 0.3] (camkii-camkii-cis-o1) {}
        node[outcome, pos = 0.55] (camkii-camkii-cis-o2) {}
        node[outcome, pos = 0.8] (camkii-camkii-cis-o3) {}
        -- (\x2,\y3)
        node[outcome, pos = 0.25] (camkii-camkii-cis-o4) {}
        node[ui, pos = 0.5] {cis}
        coordinate[anchor point, pos = 0.75] (camkii-camkii-cis-c1)
        -- (\x2,\y2)
        coordinate[anchor point] (camkii-camkii-cis-c2) at (\x2,\y4) {}
        node[outcome, pos = 0.6] (camkii-camkii-cis-o5) {}
    ;

    \path let \p1=(camkii-camkii-cis-o5), \p2=(camkii-cam-dep) in coordinate[anchor point] (camkii-cam-c1) at (\x2,\y1);

    \draw[absolute inhibition] (camkii-camkii-cis-o5) -- (camkii-cam-c1);
    \draw[absolute inhibition] (camkii-cam-o1) -- (camkii-camkii-cis-c2);
    \draw[absolute inhibition] (camkii-camkii-trans-o2) -- (camkii-cam-c2);

    \draw[assignment] (t306-varp) -- (t306)
    node[outcome, pos = 0.2] (t306-varp-o1) {}
    coordinate[anchor point, pos = 0.55] (t306-varp-c1) {}
    ;

    \draw[assignment] let \p1=(dv), \p2=(catrue) in (catrue) -- (ex)
        coordinate[anchor point] (ex-catrue-c1) at (\x2,\y1) {}
    ;

    \draw[interaction] (ca) -- (cam)
        node[outcome, pos = 0.5] (ca-cam-o1) {};
    ;

    \coordinate (camkii-glurptrue-glurp-dep) at (camkii.190);
    \draw[assignment] let \p1=(camkii-glurptrue-glurp-dep), \p2=(glurptrue) in (glurptrue) -- (glurp)
        node[outcome, pos = 0.2] (glurptrue-glurp-o1) {}
        coordinate[anchor point] (glurptrue-glurp-c1) at (\x2,\y1)
    ;

    \draw[assignment] let \p1=(glurptrue-glurp-o1), \p2=(psd) in (psd) -- (location)
        coordinate[anchor point] (psd-location-c1) at (\x2,\y1) {}
        node[outcome, pos = 0.5] (psd-location-o1) {}
    ;


    \node[or, minimum width = 10pt, minimum height = 18pt, connectors = horizontal, right = 0.5cm of t306-varp-c1] (or) {};

    \draw[inhibition] (t306-varp-o1) -| (camkii-cam-c3);

    \draw[logic arc] (camkii-camkii-trans-o1) |- (or);
    \draw[logic arc] (camkii-cam-o2) |- (or);
    \draw[absolute inhibition] (or) -- (t306-varp-c1);

    \draw[stimulation] (dv) -- (ex-catrue-c1);

    \draw[stimulation] (ca-cam-o1) -| (camkii-cam-c4);

    \draw[stimulation] (glurptrue-glurp-o1) -- (psd-location-c1);

    \node[phenotype, minimum width = 40pt, minimum height = 20pt, left = 1cm of psd-location-o1] (ltp) {LTP};

    \draw[stimulation] (psd-location-o1) -- (ltp);

    \draw[stimulation] (camkii-glurptrue-glurp-dep) -- (glurptrue-glurp-c1)
        coordinate[anchor point, pos = 0.6] (camkii-glurptrue-glurp-c1)
    ;

    \draw[absolute inhibition] (camkii-camkii-cis-o3) -| (camkii-glurptrue-glurp-c1);

    \node[delay, above left = 0.0cm and 0.5cm of camkii] (delay) {};
    \draw[logic arc] (camkii.170) -| (delay);
    \draw[necessary stimulation] (delay) |- (t306-varp-c1)
        coordinate[anchor point, pos = 0.57] (delay-t306-varp-c1)
        node[ui, pos = 0.7] {cis}
    ;

    \draw[absolute inhibition] (camkii-camkii-cis-o1) -| (delay-t306-varp-c1)
        node[ui, pos = 0.57] {cis}
    ;

    \draw[necessary stimulation] let \p1=(delay), \p2=(camkii.west), \p3=(t286true-truefalse-c2) in
        (camkii.west) -- ($(-0.7cm,0cm)+(\x1,\y2)$) -- ($(-0.7cm,-0.5cm)+(\x1,\y3)$)
        node[ui, pos = 0.8] {trans}
        -- ($(0cm,-0.5cm)+(\x2,\y3)$) -- (t286true-truefalse-c2)
    ;

    \draw[absolute inhibition] let \p1=(ppase.north), \p2=(dv.east), \p3=(cam.south), \p4=(t286true-truefalse-o1) in
    (t286true-truefalse-o1) -- ($(\x4,\y1)+(0,0.5cm)$) -- ($(\x2,\y1)+(0.5cm,0.5cm)$) -- ($(\x2,\y3)+(0.5cm,-0.5cm)$) -| (camkii-camkii-cis-c1);

    \draw[absolute inhibition] let \p1=(camkii-camkii-cis-o4), \p2=(cam.south), \p3=(ltp.west), \p4=(t286true-truefalse-c1), \p5=(camkii.west) in
    (camkii-camkii-cis-o4) -- ($(\x1,\y2)+(0,-0.5cm)$) -- ($(\x3,\y2)+(-0.5cm,-0.5cm)$) -- ($(\x3,\y4)+(-0.5cm,-0.5cm)$) -- ($(\x5,\y4)+(-0.42cm,-0.5cm)$) -- (t286true-truefalse-c1);
    ;

\end{tikzpicture}

\end{document}
